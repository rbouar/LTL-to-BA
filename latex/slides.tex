\documentclass{beamer}
\hypersetup{pdfstartview={Fit}}

\usepackage[french]{babel}
\usepackage[T1]{fontenc}
\usepackage[utf8]{inputenc}
\usepackage{eurosym}

\usepackage{amsmath}
\usepackage{amssymb}
\usepackage{amsthm}
\usepackage{mathrsfs}

% Dessiner les automates
\usepackage{tikz}
\usetikzlibrary{automata, positioning, arrows, shapes,shapes.geometric}

% quelques définitions
\theoremstyle{plain}
\newtheorem{thm}{Théorème}
\newtheorem{cor}[thm]{Corollaire}
\newtheorem{lem}[thm]{Lemme}
\newtheorem{prop}{Proposition}

\theoremstyle{definition}
\newtheorem{defi}{Définition}
\newtheorem{rmq}{Remarque}
\newtheorem{ex}{Exemple}
\newtheorem{qst}{Question}

\usetheme{Montpellier}
%\useoutertheme{infolines}
%\usecolortheme{whale}
\beamertemplatenavigationsymbolsempty

\title {Automates et logique temporelle LTL}
\author{
  Souffan Nathan \and
  Bouarah Romain \and
  \\Supervisé par François Laroussinie
}
\date{\today}

\begin{document}

\begin{frame}[plain]
  \titlepage
\end{frame}


\section{Introduction}
\subsection{Sujet}
\begin{frame}
  \frametitle{Le sujet}

  \begin{itemize}
  \item Découvrir la logique LTL
  \item Reconnaître les modèles d'une formule LTL à l'aide d'un automate
  \item Implémenter la construction de cet automate
  \end{itemize}
\end{frame}


\subsection{Vérification de modèles}
\begin{frame}
  \frametitle{Utilité}

  \begin{defi}[Vérification de modèles]
    La vérification de modèles, ou \textit{model checking}, consiste à vérifier certaines propriétés sur le modèle d'un système.
  \end{defi}
\end{frame}


\begin{frame}
  \frametitle{Un exemple basique}
  
  \begin{figure}
    \centering
    \begin{tikzpicture}[->, auto, initial text=, font=\small, node distance=5cm]
      \node[state, initial, rectangle] (q0) {a : attente d'1\euro{}};
      \node[state, right of=q0,rectangle] (q1) {e : eau de source};
      
      \draw (q0) edge[loop, above] node{} (q0);
      \draw (q0) edge[above] node{} (q1);
    \end{tikzpicture}
    \caption{Système de transition d'états pour un distributeur d'eau de source}
  \end{figure}

  Une propriété à vérifier : si la machine attend une infinité de fois 1 \euro{}, alors toujours, dans le futur, de l'eau de source est délivrée
\end{frame}


\section{LTL}
\subsection{Définitions}
\begin{frame}
  \frametitle{Syntaxe}
  Une formule LTL $f$ peut s'écrire comme :
  \begin{itemize}
  \item $p$ : atome
  \item $\top$ : tautologie
  \item $\lnot f_1$ : négation
  \item $f_1 \land f_2$ : conjonction
  \item $X f_1$ : suivant
  \item $f_1 U f_2$ : jusqu'à
  \end{itemize}
  Avec $f_1, f_2$ des formules LTL et $p \in AP$ une proposition atomique.
\end{frame}


\begin{frame}
  \frametitle{Sémantique}

  On interprète une formule sur une position $i \geq 0$ le long d'une exécution étiquetée $(p, l)$ où $p \in Q^\omega$ et $l : Q \to 2^{AP}$ 
  
  \begin{itemize}
  \item $p,l,i \models v \Leftrightarrow v \in l(p(i))$ où $v \in AP$
  \item $p,l,i \models \top$
  \item $p,l,i \models \varphi \land \psi \Leftrightarrow [(p,l,i \models \varphi) \land (p,l,i \models \psi)]$
  \item $p,l,i \models \lnot \varphi \Leftrightarrow p,l,i \not \models \varphi$
  \item $p,l,i \models X\varphi \Leftrightarrow p,l,i+1 \models \varphi$
  \item $p,l,i \models \varphi\, U \psi \Leftrightarrow [\exists j \geq i\textrm{ tel que, } p,l,i \models \psi \textrm{ et } \forall i \leq j < k, \:\: p,l,k\models \varphi]$
  \end{itemize}
\end{frame}

  
\subsection{Exemples}
\begin{frame}
  ajouter des exemples
\end{frame}


\section{Automate de Büchi}
\begin{frame}
  \begin{defi}[Automate de Büchi]
    Un automate de Büchi est un quintuplet $\mathcal{A}=(\Sigma, Q, Q_I, \Delta, \mathscr{F})$ où :
    \begin{itemize}
    \item $\Sigma$ est un alphabet
    \item $Q$ est l'ensemble des états
    \item $Q_I \subseteq Q$ est l'ensemble des états initiaux.
    \item $\Delta \subseteq Q \times \Sigma \times Q$ est l'ensemble des transitions.
    \item $\mathscr{F} \subseteq Q$ est l'ensemble des états finaux.
      Un mot $w$ est accepté s'il existe une exécution acceptante de $\mathcal{A}$ sur $w$.
    \end{itemize}
  \end{defi}
\end{frame}

\begin{frame}
  \frametitle{Un exemple d'automate de Büchi}

  Soit $\mathcal{A}=(\{a,b\}, \{q_0, q_1\}, \{q_0\}, \Delta, \{q_1\})$ un automate de Büchi.
  L'ensemble des transitions $\Delta$ est donné dans la figure.
  \begin{figure}
    \centering
    \begin{tikzpicture}[->, >=stealth, node distance=3cm, initial text=$ $, on grid]
      \node[state, initial] (q0) {$q_0$};
      \node[state, accepting, right of=q0] (q1) {$q_1$};
      \draw (q0) edge[bend left, above] node{a} (q1)
      (q1) edge[bend left, below] node{b} (q0)
      (q1) edge[loop above] node{a} (q1);
    \end{tikzpicture}
    \caption{Représentation graphique de $\mathcal{A}$}
  \end{figure}
\end{frame}


\begin{frame}[<+->]
  \begin{defi}[Exécution]
    Soit $w \in \Sigma^\omega$ un mot infini.
    Une exécution de $\mathcal{A}$ sur $w$ est une suite infinie $\rho = q_0q_1q_2\dots \in Q^\omega$ telle que :
    \[
      \forall i \geq 0 \quad (q_i, w_i, q_{i+1}) \in \Delta
    \]
  \end{defi}
  
  \begin{defi}[Exécution acceptante]
    $\rho \in Q^\omega$ une exécution de $\mathcal{A}$ est dite acceptante si :
    \[
      Etats_{\#\infty}(\rho) \cap \mathscr{F} \neq \varnothing
    \]
    où $Etats_{\#\infty}(\rho)$ est l'ensemble des états apparaissants une infinité de fois dans $\rho$.
  \end{defi}
\end{frame}


\begin{frame}
  \frametitle{Un exemple d'automate de Büchi}

  \begin{figure}
    \centering
    \begin{tikzpicture}[->, >=stealth, node distance=3cm, initial text=$ $, on grid]
      \node[state, initial] (q0) {$q_0$};
      \node[state, accepting, right of=q0] (q1) {$q_1$};
      
      \draw (q0) edge[loop, above] node{a} (q0);
      \draw (q0) edge[bend left, above] node{a} (q1);
      \draw (q1) edge[bend left, below] node{b} (q0);
      \draw (q1) edge[loop above] node{a} (q1);
    \end{tikzpicture}
    \caption{Un automate de Büchi}
  \end{figure}

  Pour le mot $a^\omega$, on a :
  \begin{itemize}
  \item $q_0q_0\dots$ qui est une exécution
    \pause
  \item $q_0q_1\dots$ qui est une exécution acceptante
    \pause
  \item ...
  \end{itemize}
\end{frame}


% \begin{defi}[Langage reconnu]
%   Le langage reconnu par l'automate, noté $\mathcal{L}(\mathcal{A})$, est l'ensemble des mots $w$ tel qu'il existe une exécution acceptante de $w$ sur $\mathcal{A}$.
% \end{defi}

\section{Formule LTL -> automate de Büchi}
\begin{frame}
  
\end{frame}

\section{Présentation du prototype}
\begin{frame}
  
\end{frame}

\end{document}
