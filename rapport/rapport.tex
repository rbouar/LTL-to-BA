\documentclass[12pt,a4paper]{article}

\usepackage[french]{babel}
\usepackage[T1]{fontenc}
\usepackage{amssymb}


\title {Automates et logique temporelle LTL}
\author{
  Souffan Nathan \and
  Bouarah Romain \and
  \\Supervisé par François Laroussinie
}


\begin{document}
\maketitle
\newpage

\section{Logique temporelle}

Si pour la logique une propriété ne peut être que vrai ou fausse, celle ci peut être vrai à un certain moment puis fausse par la suite en logique temporelle. On peut considérer plusieurs espace temps. On peut prendre le temps comme étant $\mathbb{R}$, $\mathbb{Z}$ ou encore comme ce sera le cas par la suite $\mathbb{N}$. Celle ci s'appelle la logique \underline{LTL} (Logique Temporelle Linéaire).

\subsection{LTL}
On se place donc dans le cas d'une logique temporelle linéaire, On défini $AP$ l'ensemble des propositions atomiques. Il y a alors une valuation des $AP$ pour chaque $n$ dans $\mathbb{N}$.

\subsubsection{Syntaxe}
\subsubsection{Sémantique}
Les formules sont interprétés sur une position $i \geq 0$ le long d'une exécution étiqueté $(p, l)$ où $p \in Q^{\omega}$ et $l : Q \to 2^{AP}$. \\
On note ainsi $p, l, i \models \varphi$ le fait que $\varphi$ est vraie en $i$ le long de $(p, l)$ et on a: 
\begin{itemize}
	\item $p, l, i \models P \Leftrightarrow P \in l(p(i))$ 
	\item $p, l, i \models \varphi \land \psi \Leftrightarrow [(p,l,i \models \varphi) \textrm{ et } (p,l,i \models \psi)]$
	\item $p,l,i \models \varphi \lor \psi \Leftrightarrow [(p,l,i \models \varphi)\textrm{ ou } (p,l,i \models \psi)]$
	\item $p,l,i \models \lnot \varphi \Leftrightarrow p,l,i \not \models \varphi$
	\item $p,l,i \models X\varphi \Leftrightarrow p,l,i+1 \models \varphi$
	\item $p,l,i \models F\varphi \Leftrightarrow [\exists j \geq i, p,l,j \models \varphi]$
	\item $p,l,i \models \varphi\, U \psi \Leftrightarrow [\exists j \geq i, \:\: p,l,i \models \varphi \textrm{ et } \forall i \leq j \leq k, \:\: p,l,i\models \varphi]$
	\item $p,l,i \models G\varphi \Leftrightarrow p,l,i \models \lnot F\lnot \varphi$
	
\end{itemize}
On ajoute à cela, $\varphi \equiv \psi \Leftrightarrow [p,l,i \models \varphi \Leftrightarrow p,l,i \models \psi]$\\
\vspace*{2mm}
\underline{Exemples}: \\
$a, b \in AP$
\begin{itemize}
	\item $GFa$: (toujours(futur $a$)) ce qui signifie il y a une infinité de positions où $a$ est vrai.
	\item $aU(Gb)$: $a$ est vrai tant que $b$ est faux, dès que $a$ est faux, $b$ est toujours vrai par la suite
	\item $(a\lor b ) U a \equiv G(a \lor b)$ car la première formule signifie on a $a$ dès qu'on à pas $a\lor b$ donc on doit toujours avoir $a \lor b$
	
\end{itemize}




\end{document}