\documentclass[12pt,a4paper]{article}

\usepackage[colorlinks]{hyperref}
\usepackage[french]{babel}
\usepackage[T1]{fontenc}

\usepackage{amsmath}
\usepackage{amssymb}
\usepackage{amsthm}
\usepackage{mathrsfs}

\usepackage[nameinlink]{cleveref}

\usepackage{tikz}
\usetikzlibrary{automata, positioning, arrows, shapes}

\usepackage{enumerate} 


% quelques définitions
\theoremstyle{plain}
\newtheorem{thm}{Théorème}
\newtheorem{cor}[thm]{Corollaire}
\newtheorem{lem}[thm]{Lemme}
\newtheorem{prop}{Proposition}
\newtheorem{dem}{Démonstration}

\theoremstyle{definition}
\newtheorem{defi}{Définition}
\newtheorem{rmq}{Remarque}
\newtheorem{ex}{Exemple}

\title {Automates et logique temporelle LTL}
\author{
  Souffan Nathan \and
  Bouarah Romain \and
  \\Supervisé par François Laroussinie
}


\begin{document}
\maketitle
\tableofcontents

\section{Automate de Büchi}
Les automates de Büchi sont un type particulier d'automate sur les mots infinis.
Les automates sur les mots infinis (ou $\omega$-automates) sont des automates finis qui acceptent des mots infinis.

\begin{defi}[Automate de Büchi]
  Un automate de Büchi est un quintuplet $\mathcal{A}=(\Sigma, Q, Q_I, \Delta, \mathscr{F})$ où :
  \begin{itemize}
  \item $\Sigma$ est un ensemble fini appelé alphabet de $\mathcal{A}$.
  \item $Q$ est un ensemble fini. Les éléments de $Q$ sont les états de $\mathcal{A}$.
  \item $Q_I \subseteq Q$ est l'ensemble des états initiaux.
  \item $\Delta \subset Q \times \Sigma \times Q$ est l'ensemble des transitions.
  \item $\mathscr{F} \subseteq Q$ est l'ensemble des états finaux (ou états acceptants).
    Un mot $w$ est accepté s'il existe une exécution acceptante de $\mathcal{A}$ sur $w$.
  \end{itemize}
\end{defi}

\begin{defi}[Exécution]
  Soient $w \in \Sigma^\omega$ un mot infini et $\mathcal{A}=(\Sigma, Q, Q_I, \Delta, \mathscr{F})$ un automate de Büchi.
  Une exécution de $\mathcal{A}$ sur $w$ est une suite infinie $\rho = q_0q_1q_2\dots \in Q^\omega$ telle que :
  \[
    \forall i \geq 0 \quad (q_i, w_i, q_{i+1}) \in \Delta
  \]
\end{defi}

\begin{defi}[Exécution acceptante]
  \label{exec-accept}
  Soient $\mathcal{A}=(\Sigma, Q, Q_I, \Delta, \mathscr{F})$ un automate de Büchi et $\rho \in Q^\omega$ une exécution de $\mathcal{A}$.
  On dit que $\rho$ est une exécution acceptante si :
  \[
    Etats_{\#\infty}(\rho) \cap \mathscr{F} \neq \varnothing
  \]
  où $Etats_{\#\infty}(\rho)$ est l'ensemble des états apparaissants une infinité de fois dans $\rho$.
\end{defi}

\begin{defi}[Langage reconnu]
  Soit $\mathcal{A}$ un automate.
  Le langage reconnu par l'automate, noté $\mathcal{L}(\mathcal{A})$, est l'ensemble des mots $w$ tel qu'il existe une exécution acceptante de $w$ sur $\mathcal{A}$.
\end{defi}

\begin{ex}
  Soit $\mathcal{A}=(\{a,b\}, \{q_0, q_1\}, \{q_0\}, \Delta, \{q_1\})$ un automate de Büchi.
  L'ensemble des transitions $\Delta$ est donné dans la figure.
  \begin{figure}[h]
    \centering
    \begin{tikzpicture}[->, >=stealth, node distance=3cm, initial text=$ $, on grid]
      \node[state, initial] (q0) {$q_0$};
      \node[state, accepting, right of=q0] (q1) {$q_1$};
      \draw (q0) edge[bend left, above] node{a} (q1)
      (q1) edge[bend left, below] node{b} (q0)
      (q1) edge[loop above] node{a} (q1);
    \end{tikzpicture}    
    \caption{Automate de Büchi}
  \end{figure}
  
  Cet automate reconnaît le mot $w = aaa\dots$ car pour l'exécution $\rho = q_0q_1^\omega$ (c'est la seule) de $\mathcal{A}$ sur $w$, l'état $q_1$ apparaît une infinité de fois dans $\rho$.
  En fait, le langage reconnu (ou accepté) est $aa^\omega | a(a^*ba)^\omega$.
\end{ex}

Les automates de Büchi généralisés sont une variante des automates de Büchi.
La différence se situe sur la condition d'acceptation.
\begin{defi}[Automate de Büchi généralisé]
  Un automate de Büchi généralisé est un quintuplet $\mathcal{A}=(\Sigma, Q, Q_I, \Delta, \mathscr{F})$ où :
  \begin{itemize}
  \item $\Sigma, Q, Q_I, \Delta$ sont comme précédemment.
  \item $\mathscr{F} \subseteq \mathcal{P}(Q)$ est la condition d'acceptation.
    $\mathscr{F}$ est un ensemble d'ensembles finaux. De même, un mot $w$ est accepté s'il existe une exécution acceptante de $\mathcal{A}$ sur $w$.
  \end{itemize}
  
\end{defi}
Pour un automate de Büchi généralisé, une exécution $\rho$ de $\mathcal{A}$ est acceptante si :
\[
  \forall F \in \mathscr{F} \quad Etats_{\#\infty}(\rho) \cap F \neq \varnothing
\]

\subsection{Traduction de formule LTL en automate de Büchi}
On souhaite ``traduire'' une formule LTL en un automate de Büchi. Plus précisément, on veut construire un automate qui reconnait les modèles de $\varphi$.

\begin{defi}
  On note $SubF(\varphi)$ l'ensemble des sous formules de $\varphi$ et leur négation.  
\end{defi}

\begin{ex}
  Si $\varphi = a U b$ alors $SubF(\varphi) = \{ a, \lnot a, b, \lnot b, a U b, \lnot(a U b)\}$.
\end{ex}

\begin{defi}[Sous-ensemble cohérent]
  $q \in 2^{SubF(\varphi)}$ est cohérent si toutes les conditions suivantes sont vérifiées :
  \begin{enumerate}[(i)]
  \item $\bot \not \in q$.
  \item Si $\psi_1 \land \psi_2 \in q$ alors $\psi_1 \in q$ et $\psi_2 \in q$.
  \item Si $\psi_1 \lor \psi_2 \in q$ alors $\psi_1 \in q$ ou $\psi_2 \in q$.
  \item $\psi \in q \iff \lnot \psi \not\in q$.
  \end{enumerate}
\end{defi}

\begin{rmq}
  En utilisant les lois de Morgan, on en déduit aussi que :
  \begin{enumerate}[(i)]
  \item Si $\lnot (\psi_1 \land \psi_2) \in q$ alors $\psi_1 \not \in q$ ou $\psi_2 \not \in q$.
  \item Si $\lnot (\psi_1 \lor \psi_2) \in q$ alors $\psi_1 \not \in q$ et $\psi_2 \not \in q$.
  \end{enumerate}
\end{rmq}


\begin{defi}[Sous-ensemble maximal]
  $q \in 2^{SubF(\varphi)}$ est maximal si pour tout $\psi \in SubF(\varphi)$ on a soit $\psi \in q$ soit $\lnot \psi \in q$.  
\end{defi}

\begin{defi}[Sous-ensemble conforme à la sémantique de LTL]
  \label{ss-ens-ok-ltl}
  $q \in 2^{SubF(\varphi)}$ est conforme à la sémantique de LTL :
  \begin{enumerate}[(i)]
  \item Si $\psi_1 U \psi_2 \in q$ alors on a soit $\psi_1 \in q$ soit $\psi_2 \in q$.
  \item $\forall \psi_1 U \psi_2 \in SubF(\varphi)$ si $\psi_2 \in q$ alors $\psi_1 U \psi_2 \in q$.
  \end{enumerate}
\end{defi}

\begin{ex}
  Soit $\varphi = a U (Xb)$ alors
  \[
    SubF(\varphi) = \{a, \lnot a, b, \lnot b, Xb, \lnot (Xb), a U (Xb), \lnot (aU(Xb))\}
  \]
  \begin{enumerate}
  \item $q_1 = \{\lnot a, b, Xb, a U (Xb)\}$ est un sous-ensemble cohérent, maximal et conforme à la sémantique de LTL.
  \end{enumerate}
\end{ex}

\begin{defi}
  Soient $AP$ l'ensemble des propositions atomiques et $\varphi$ une formule LTL sur $AP$.
  L'automate de Büchi généralisé pour $\varphi$ sur $AP$ est donné par $\mathcal{A}_\varphi = (2^{AP}, Q, Q_I, \Delta, \mathscr{F})$ où :
  \begin{itemize}
  \item $Q \subseteq 2^{SubF(\varphi)}$ contient tout les sous-ensembles de $SubF(\varphi)$ qui sont cohérents, maximaux et conformes à la sémantique de LTL.
  \item $Q_I = \{ q \in Q | \varphi \in q \}$. Autrement dit, tous les états contenant exactement notre formule de départ $\varphi$ sont des états initiaux.
  \item $\Delta$ est l'ensemble des transitions $(q, a, q')$ avec $q, q' \in Q$ et $a \in 2^{AP}$ vérifiant :
    \begin{enumerate}[(i)]
    \item $\forall p \in AP \quad p \in q \iff p \in a$ (i.e. $a$ possède toutes les propositions atomiques de $q$)
    \item $\forall X\psi \in SubF(\varphi) \quad X\psi \in q \iff \psi \in q'$
    \item $\forall \psi_1 U \psi_2 \in SubF(\varphi) \quad \psi_1 U \psi_2 \in q \iff \left( \psi_2 \in q \lor (\psi_1 \in q \land \psi_1 U \psi_2 \in q')\right)$
    \item $\forall \psi_1 \lor \psi_2 \in SubF(\varphi) \quad \psi_1 \lor \psi_2 \in q \iff (\psi_1 \in q' \lor \psi_2 \in q')$
    \item $\forall \psi_1 \land \psi_2 \in SubF(\varphi) \quad \psi_1 \land \psi_2 \in q \iff (\psi_1 \in q' \land \psi_2 \in q')$
    \end{enumerate}
  \item $\mathscr{F} = \{F_{\psi_1 U \psi_2} | \psi_1 U \psi_2 \in SubF(\varphi)\}$ où
    \[
      F_{\psi_1 U \psi_2} = \{q \in Q | \psi_1 U \psi_2 \not \in q \lor \psi_2\in q \}
    \]
  \end{itemize}
\end{defi}

L'idée dérrière cette construction est de faire en sorte que l'automate devine quelles sont les sous-formules de $\varphi$ vraies lorsqu'on lit un mot.

% \begin{ex}
%   Avec $\varphi = Xa$ on obtient l'automate suivant :
%   \begin{figure}[h]
%     \centering
%     \begin{tikzpicture}[->, >=stealth, node distance=3cm, initial text=$ $, on grid]
%       \tikzstyle{every node}=[rectangle, align=center]
      
%       \node[state, initial] (q0) {$\{a, Xa\}$};
%       \node[state, initial, below = of q0] (q1) at (0, -1) {$\{\lnot a, Xa\}$};
%       \node[state, right = of q0] (q3) {$\{a, \lnot Xa\}$};
%       \node[state, below = of q3, right = of q1] (q4) {$\{\lnot a, \lnot Xa\}$};

      
%     \end{tikzpicture}    
%     \caption{Automate de Büchi}
%   \end{figure}
% \end{ex}

\begin{thm}
  Soient :
  \begin{itemize}
  \item $AP$ un ensemble de propositions atomiques.
  \item $\varphi$ une formule LTL sur $AP$.
  \item $w \in (2^{AP})^\omega$ un mot infini sur l'alphabet $2^{AP}$ tel que $w, 0 \models \varphi$.
  \item $\mathcal{A}_\varphi$ l'automate de Büchi généralisé pour $\varphi$ sur $AP$.
  \end{itemize}
  Alors $w \in \mathcal{L}(\mathcal{A}_\varphi)$. 
\end{thm}

\begin{proof}
  On veut montrer que $w \in \mathcal{L}(\mathcal{A}_\varphi)$.
  D'après la \cref{exec-accept}, cela revient à montrer qu'il existe une exécution acceptante de $w$ sur $\mathcal{A}_\varphi$.

  On pose $\forall i \geq 0 \quad q_i = \{ \psi \in SubF(\varphi) | w, i \models \psi\}$ et $\rho = q_0q_1q_2\dots$
  Montrons que $\rho$ est une exécution acceptante sur $w$ dans $\mathcal{A}_\varphi$.
  \begin{itemize}
  \item $w, 0 \models \varphi$ donc
    $\varphi \in q_0$ donc
    $q_0$ est bien un état initial.
  \item Il y a bien des transition $(q_i, w_i, q_{i+1})$ dans $\mathcal{A}_\varphi$.
  \item $\rho$ vérifie bien la condition d'acceptation.
    En effet, si $\psi_1 U \psi_2 \in q_i$, alors $w, i \models \psi_1 U \psi_2$ (par construction des $q_i$)
    donc $\exists j \geq i$ tel que $w,j \models \psi_2$ et $\forall k, i \leq k \leq j \quad w, k \models \psi_1$.
    Enfin, d'après la \cref{ss-ens-ok-ltl}, on a aussi $\psi_1 U \psi_2 \in q_j$ et il existe un chemin valide jusqu'à $q_j$ ainsi $\rho$ passe infiniment souvent par $F_{\psi_1 U \psi_2}$.
  \end{itemize}
\end{proof}

\end{document}